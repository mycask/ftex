\section{Variables, Types, and Declarations}

\subsection*{Variable Names}

Variable names in Fortran consist of 1-6 characters chosen from the
letters \inlinecode{a-z} and the digits \inlinecode{0-9}. The first
character must be a letter. Fortran 77 does not distinguish between
upper and lower case, in fact, it assumes all input is upper case.
However, nearly all Fortran 77 compilers will accept lower case. If you
should ever encounter a Fortran 77 compiler that insists on upper case
it is usually easy to convert the source code to all upper case.

The words which make up the Fortran language are called \textit{reserved
words} and cannot be used as names of variable. Some of the reserved
words which we have seen so far are ``\inlinefortranss{program}'',
``\inlinefortranss{real}'', ``\inlinefortranss{stop}'' and
``\inlinefortranss{end}''.


\subsection*{Types and Declarations}

Every variable \textit{should} be defined in a \textit{declaration}.
This establishes the \textit{type} of the variable. The most common
declarations are:

\begin{fortran77}
      integer           <<list of variables>>
      real              <<list of variables>>
      double precision  <<list of variables>>
      complex           <<list of variables>>
      logical           <<list of variables>>
      character         <<list of variables>>
\end{fortran77}

The \textit{list of variables} should consist of variable names
separated by commas. Each variable should be declared exactly once. If a
variable is undeclared, Fortran 77 uses a set of \textit{implicit rules}
to establish the type. This means all variables starting with the
letters \inlinefortranss{i}-\inlinefortranss{n} are integers and all
others are real. Many old Fortran 77 programs uses these implicit rules,
but \textit{you should not}! The probability of errors in your program
grows dramatically if you do not consistently declare your variables.


\subsection*{Integers and Floating Point Variables}

Fortran 77 has only one type for integer variables. Integers are usually
stored as 32 bits (4 bytes) variables. Therefore, all integer variables
should take on values in the range \( [-m,m] \) where \( m \) is
approximately \(2 \times 10^9 \).

Fortran 77 has two different types for floating point variables, called
\inlinefortranss{real} and \inlinefortranss{double precision}. While
\inlinefortranss{real} is often adequate, some numerical calculations
need very high precision and \\
\inlinefortranss{double precision} should be used. Usually a
\inlinefortranss{real} is a 4 byte variable and the
\inlinefortranss{double precision} is 8 bytes, but this is machine
dependent. Some non-standard Fortran versions use the syntax
\inlinefortranss{real*8} to denote 8 byte floating point variables.


\subsection*{The \inlinefortranss{parameter} Statement}

Some constants appear many times in a program. It is then often
desirable to define them only once, in the beginning of the program.
This is what the \inlinefortranss{parameter} statement is for. It also
makes programs more readable. For example, the circle area program
should rather have been written like this:

\begin{fortran77}
      program circle
      real r, area, pi
      parameter (pi = 3.14159)
 
c This program reads a real number r and prints
c the area of a circle with radius r.
 
      write (*,*) 'Give radius r:'
      read  (*,*) r
      area = pi*r*r
      write (*,*) 'Area = ', area
 
      stop
      end
\end{fortran77}

The syntax of the \inlinefortranss{parameter} statement is

\ifdefined\mini
\begin{fortran77}
      parameter (<<name>> = <<constant>>, <<...>> ,
     +           <<name>> = <<constant>>)
\end{fortran77}
\else
\begin{fortran77}
      parameter (<<name>> = <<constant>>, <<...>> , <<name>> = <<constant>>)
\end{fortran77}
\fi

The rules for the \inlinefortranss{parameter} statement are:

\begin{itemize}
    \item The \inlinefortranss{<<name>>} defined in the parameter
        statement is not a variable but rather a constant. (You cannot
        change its value at a later point in the program.)

    \item A \inlinefortranss{<<name>>} can appear in at most one
        \inlinefortranss{parameter} statement.

    \item The \inlinefortranss{parameter} statement(s) must come before
        the first executable statement.
\end{itemize}

Some good reasons to use the \inlinefortranss{parameter} statement are:

\begin{itemize}
    \item It helps reduce the number of typos.

    \item It makes it easier to change a constant that appears many
        times in a program.

    \item It increases the readability of your program.
\end{itemize}
