\section{Expressions and Assignment}

\subsection*{Constants}

The simplest form of an expression is a \textit{constant}. There are 6
types of constants, corresponding to the 6 data types. Here are some
\inlinefortranss{integer} constants:

\begin{code}
      1
      0
      -100
      32767
      +15
\end{code}

Then we have real constants:

\begin{code}
      1.0
      -0.25
      2.0E6
      3.333E-1
\end{code}

The \inlinecode{E}-notation means that you should multiply the constant
by 10 raised to the power following the ``\inlinecode{E}''. Hence,
\inlinecode{2.0E6} is two million, while \inlinecode{3.333E-1} is
approximately one third.

For constants that are larger than the largest \inlinefortranss{real}
allowed, or that requires high precision, \inlinefortranss{double
precision} should be used. The notation is the same as for
\inlinefortranss{real} constants except the ``\inlinecode{E}'' is
replaced by a ``\inlinecode{D}''. Examples:

\begin{code}
      2.0D-1
      1D99
\end{code}

Here \inlinecode{2.0D-1} is a \inlinefortranss{double precision}
one-fifth, while \inlinecode{1D99} is a one followed by 99 zeros.

The next type is \inlinekeyword{complex} constants. This is designated
by a pair of constants (integer or real), separated by a comma and
enclosed in parentheses. Examples are:

\begin{code}
      (2, -3)
      (1., 9.9E-1)
\end{code}

The first number denotes the real part and the second the imaginary part.

The fifth type is \inlinefortranss{logical} constants. These can only
have one of two values:

\begin{fortran77}
      .TRUE.
      .FALSE.
\end{fortran77}

Note that the dots enclosing the letters are required.

The last type is \inlinekeyword{character} constants. These are most
often used as an \textit{array} of characters, called a \textit{string}.
These consist of an arbitrary sequence of characters enclosed in
apostrophes (single quotes):

\begin{fortran77}
      'ABC'
      'Anything goes!'
      'It is a nice day'
\end{fortran77}

Strings and \inlinekeyword{character} constants are case sensitive. A
problem arises if you want to have an apostrophe in the string itself.
In this case, you should double the apostrophe:

\begin{fortran77}
      'It''s a nice day'
\end{fortran77}


\subsection*{Expressions}

The simplest non-constant expressions are of the form

\begin{fortran77}
      <<operand>> <<operator>> <<operand>>
\end{fortran77}

and an example is

\begin{fortran77}
      x + y
\end{fortran77}

The result of an expression is itself an operand, hence we can nest
expressions together like

\begin{fortran77}
      x + 2 ** y
\end{fortran77}

This raises the question of precedence: Does the last expression mean
\inlinefortranss{x + (2*y)} or \inlinefortranss{(x+2)*y}? The precedence
of arithmetic operators in Fortran 77 are (from highest to lowest):

\begin{fortran77}
      **   <<{exponentiation}>>
      *,/  <<{multiplication, division}>>
      +,-  <<{addition, subtraction}>>
\end{fortran77}

All these operators are calculated left-to-right, except the
exponentiation operator \inlinecode{**}, which has right-to-left
precedence. If you want to change the default evaluation order, you can
use parentheses.

The above operators are all binary operators. there is also the unary
operator \inlinefortranss{-} for negation, which takes precedence over
the others. Hence an expression like \inlinefortranss{-x+y} means what
you would expect.

Extreme caution must be taken when using the division operator, which
has a quite different meaning for integers and reals. If the operands
are both integers, an integer division is performed, otherwise a real
arithmetic division is performed. For example, \inlinefortranss{3/2}
equals \inlinefortranss{1}, while \inlinefortranss{3./2.} equals
\inlinefortranss{1.5} (note the decimal points).


\subsection*{Assignment}

The assignment has the form

\begin{fortran77}
      <<variable_name>> = <<expression>>
\end{fortran77}

The interpretation is as follows: Evaluate the right hand side and
assign the resulting value to the variable on the left. The expression
on the right may contain other variables, but these never change value!
For example,

\begin{fortran77}
      area = pi * r**2
\end{fortran77}

does not change the value of \inlinefortranss{pi} or
\inlinefortranss{r}, only \inlinefortranss{area}.


\subsection*{Type Conversion}

When different data types occur in the same expression, \textit{type
conversion} has to take place, either explicitly or implicitly. Fortran
will do some type conversion implicitly. For example,

\begin{fortran77}
      real x
      x = x + 1
\end{fortran77}

will convert the \inlinefortranss{integer} one to the
\inlinefortranss{real} number one, and has the desired effect of
incrementing \inlinefortranss{x} by one. However, in more complicated
expressions, it is good programming practice to force the necessary type
conversions explicitly. For numbers, the following functions are
available:

\begin{fortran77}
      int
      real
      dble
      ichar
      char
\end{fortran77}

The first three have the obvious meaning. \inlinefortranss{ichar} takes
a \inlinekeyword{character} and converts it to an
\inlinefortranss{integer}, while \inlinekeyword{char} does exactly the
opposite.

Example: How to multiply two \inlinefortranss{real} variables
\inlinefortranss{x} and \inlinefortranss{y} using
\inlinefortranss{double precision} and store the result in the
\inlinefortranss{double precision} variable \inlinefortranss{w}:

\begin{fortran77}
      w = dble(x)*dble(y)
\end{fortran77}

Note that this is different from

\begin{fortran77}
      w = dble(x*y)
\end{fortran77}
