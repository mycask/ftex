\section{Fortran 77 Basics}

A Fortran program is just a sequence of lines of text. The text has to
follow a certain structure to be a valid Fortran program. We start by
looking at a simple example:

\begin{fssl}
      program circle
      real r, area

c This program reads a real number r and prints
c the area of a circle with radius r.

      write (*,*) 'Give radius r:'
      read  (*,*) r
      area = 3.14159*r*r
      write (*,*) 'Area = ', area

      stop
      end
\end{fssl}

The lines that begin with with a ``\fssi{c}'' are \textit{comments} and
have no purpose other than to make the program more readable for humans.
Originally, all Fortran programs had to be written in all upper-case
letters. Most people now write lower-case since this is more legible,
and so will we. You may wish to mix case, but Fortran is not
case-sensitive, so ``\fssi{X}'' and ``\fssi{x}'' are the same variable.


\section*{Program Organization}

A Fortran program generally consists of a main program (or driver) and
possibly several subprograms (procedures or subroutines). For now we
will place all the statements in the main program; subprograms will be
treated later. The structure of a main program is:

\begin{fssl}
      program <<(\textit{name})>>

      <<(\textit{declarations})>>

      <<(\textit{statements})>>

      stop
      end
\end{fssl}


In this tutorial, words that are in \textit{italics} should not be taken
as literal text, but rather as a description of what belongs in their
place.

The \fssi{stop} statement is optional and may seem superfluous since the
program will stop when it reaches the end anyway, but it is recommended
to always terminate a program with the stop statement to emphasize that
the execution flow stops there.

You should note that you cannot have a variable with the same name as
the program.


\section*{Column Position Rules}

Fortran 77 is \textit{not} a free-format language, but has a very strict
set of rules for how the source code should be formatted. The most
important rules are the column position rules:

\begin{plainl}
Col. 1    : Blank, or a "c" or "*" for comments
Col. 1-5  : Statement label (optional)
Col. 6    : Continuation of previous line (optional)
Col. 7-72 : Statements
Col. 73-80: Sequence number (optional, rarely used today)
\end{plainl}

Most lines in a Fortran 77 program starts with 6 blanks and ends before column 72, i.e. only the statement field is used.
Comments

A line that begins with the letter "c" or an asterisk in the first column is a comment. Comments may appear anywhere in the program. Well-written comments are crucial to program readability. Commercial Fortran codes often contain about 50% comments. You may also encounter Fortran programs that use the exclamation mark (!) for comments. This is not a standard part of Fortran 77, but is supported by several Fortran 77 compilers and is explicitly allowed in Fortran 90. When understood, the exclamation mark may appear anywhere on a line (except in positions 2-6).
Continuation

Sometimes, a statement does not fit into the 66 available columns of a single line. One can then break the statement into two or more lines, and use the continuation mark in position 6. Example:

c23456789 (This demonstrates column position!)

c The next statement goes over two physical lines
      area = 3.14159265358979
     +       * r * r

Any character can be used instead of the plus sign as a continuation character. It is considered good programming style to use either the plus sign, an ampersand, or digits (using 2 for the second line, 3 for the third, and so on).
Blank spaces

Blank spaces are ignored in Fortran 77. So if you remove all blanks in a Fortran 77 program, the program is still acceptable to a complier but almost unreadable to humans.

\footer
