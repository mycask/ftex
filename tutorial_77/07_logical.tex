\section{Logical Expressions}

Logical expressions can only have the value \inlinefortranss{.TRUE.} or
\inlinefortranss{.FALSE.}. A logical expression can be formed by
comparing arithmetic expressions using the following \textit{relational
operators}:

\begin{fortran77}
      .LT.  <<meaning <>>
      .LE.          <<<=>>
      .GT.          <<>>>
      .GE.          <<>=>>
      .EQ.          <<=>>
      .NE.          <</=>>
\end{fortran77}

So you \textit{cannot} use symbols like \inlinefortranss{<} or
\inlinefortranss{=} for comparison in Fortran 77, but you have to use
the correct two-letter abbreviation enclosed by dots!

Logical expressions can be combined by the \textit{logical operators}
\inlinefortranss{.AND.}, \inlinefortranss{.OR.}, and
\inlinefortranss{.NOT.} which have the obvious meaning.


\subsection*{Logical Variables and Assignment}

Truth values can be stored in \textit{logical variables}. The assignment
is analogous to the arithmetic assignment. Example:

\begin{fortran77}
      logical a, b
      a = .TRUE.
      b = a .AND. 3 .LT. 5/2
\end{fortran77}

The order of precedence is important, as the last example shows. The
rule is that arithmetic expressions are evaluated first, then relational
operators, and finally logical operators. Hence \inlinefortranss{b} will
be assigned \inlinefortranss{.FALSE.} in the example above. Among the
logical operators the precedence (in the absence of parenthesis) is that
\inlinefortranss{.NOT.} is done first, then \inlinefortranss{.AND.},
then \inlinefortranss{.OR.} is done last.

Logical variables are seldom used in Fortran. But logical expressions
are frequently used in conditional statements like the
\inlinefortranss{if} statement.
