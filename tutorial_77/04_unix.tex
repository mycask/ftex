\section{How to Use Fortran on the Unix Computers at Stanford}

\subsection*{Practical Details}

For the course ME390 you will need an account on the \textit{leland}
computers. If you don't have one, you should go over to the consultants'
desk on the second floor of Sweet Hall and open an account ASAP.

In this class, we will be using Fortran under the Unix operating system.
If you have no previous experience with Unix, you will have to learn the
basics on your own. The consultants at Sweet Hall have some brochures
that can be useful. There is also a class taught in the CS department
that gives an introduction to Unix.

You can use any Unix workstation that has a Fortran 77 compiler. We
recommend that you use one of the Sun workstations, so you should log
onto one of the ``elaine'', ``epic'', ``adelbert'' or ``saga'' machines.


\subsection*{Source Code, Object Code, Compiling, and Linking}

A Fortran program consists of plain text that follows certain rules
(syntax). This is called the \textit{source code}. You need to use an
\textit{editor} to write (edit) the source code. The most common editors
in Unix are \inlinecode{emacs} and \inlinecode{vi}, but these can be a
bit tricky for novice users. You may want to use a simpler editor, like
\inlinecode{xedit} which runs under X-windows.

When you have written a Fortran program, you should save it in a file
that has the extension \inlinecode{.f} or \inlinecode{.for}. Before you
can execute the program, you have to translate the program into machine
readable form. This is done by a special program called a
\textit{compiler}. The Fortran 77 compilers are usually called
\inlinecode{f77}. The output from the compilation is given the somewhat
cryptic name \inlinecode{a.out} by default, but you can choose another
name if you wish. To run the program, simply type the name of the
executable file, for example \inlinecode{a.out}. (This explanation is a
bit oversimplified. Really the compiler translates source code into
\textit{object code} and the linker/loader makes this into an
\textit{executable}.)


\subsubsection*{Examples}

In the class directory there is a small Fortran program called
\inlinecode{circle.f}. You can compile and run it by following these
steps:

\begin{code}
    cp /usr/class/me390/circle*.f .
    f77 circle.f
    a.out
\end{code}

(Note that there are several dots (periods) there which can be easy to
miss!) If you need to have several executables at the same time, it is a
good idea to give the executables descriptive names. This can be
accomplished using the \inlinecode{-o} option. For example,

\begin{code}
    f77 circle.f -o circle.out
\end{code}

will compile the file \inlinecode{circle.f} and save the executable in
the file \inlinecode{circle.out}. Please note that object codes and
executables take a lot of disk space, so you should delete them when you
are not using them. (The remove command in Unix is \inlinecode{rm}.)

In the previous examples, we have not distinguished between
\textit{compiling} and \textit{linking}. These are two different
processes but the Fortran compiler performs them both, so the user
usually does not need to know about it. But in the next example we will
use \textit{two} source code files.

\begin{code}
    f77 circle1.f circle2.f
\end{code}

This will generate \textit{three} files, the two object code files
\inlinecode{circle1.o} and \inlinecode{circle2.o}, plus the executable
file \inlinecode{a.out}. What really happened here, is that the Fortran
compiler first compiled each of the source code files into object files
(ending in \inlinecode{.o}) and then linked the two object files
together into the executable \inlinecode{a.out}. You can separate these
two steps by using the \inlinecode{-c} option to tell the compiler to
only compile the source files:

\begin{code}
    f77 -c circle1.f circle2.f
    f77 circle1.o circle2.o
\end{code}

Compiling separate files like this may be useful if there are many files
and only a few of them needs to be recompiled. In Unix there is a useful
command called \inlinecode{make} which is usually used to handle large
software packages with many source files. These packages come with a
\inlinecode{makefile} and all the user has to do is to type
\inlinecode{make}. Writing makefiles is a bit complicated so we will not
discuss this in this tutorial.

\footer
