\section{What is Fortran?}

Fortran is a general purpose programming language, mainly intended for
mathematical computations in e.g. engineering. Fortran is an acronym for
FORmula TRANslation, and was originally capitalized as FORTRAN. However,
following the current trend to only capitalize the first letter in
acronyms, we will call it Fortran. Fortran was the first ever high-level
programming language. The work on Fortran started in the 1950's at IBM
and there have been many versions since. By convention, a Fortran
version is denoted by the last two digits of the year the standard was
proposed. Thus we have

\begin{itemize}
    \item Fortran 66
    \item Fortran 77
    \item Fortran 90 (95)
\end{itemize}

The most common Fortran version today is still Fortran 77, although
Fortran 90 is growing in popularity. Fortran 95 is a revised version of
Fortran 90 which (as of early 1996) is expected to be approved by ANSI
soon. There are also several versions of Fortran aimed at parallel
computers. The most important one is High Performance Fortran (HPF),
which is a de-facto standard.

Users should be aware that most Fortran 77 compilers allow a superset of
Fortran 77, i.e. they allow non-standard extensions. In this tutorial we
will emphasize standard ANSI Fortran 77.


\subsection*{Why Learn Fortran?}

Fortran is the dominant programming language used in engineering
applications. It is therefore important for engineering graduates to be
able to read and modify Fortran code. From time to time, so-called
experts predict that Fortran will rapidly fade in popularity and soon
become extinct. These predictions have always failed. Fortran is the
most enduring computer programming language in history. One of the main
reasons Fortran has survived and will survive is \textit{software
inertia}. Once a company has spent many man-years and perhaps millions
of dollars on a software product, it is unlikely to try to translate the
software to a different language. Reliable software translation is a
very difficult task.


\subsection*{Portability}

A major advantage Fortran has is that it is standardized by ANSI and ISO
(see footnotes). Consequently, if your program is written in ANSI
Fortran 77, using nothing outside the standard, then it will run on any
computer that has a Fortran 77 compiler. Thus, Fortran programs are
portable across machine platforms. (If you want to read some Fortran
Standards Documents, click
\href{http://www.fortran.com/fortran/stds_docs.html}{here}.)

\textbf{Footnotes: \\
ANSI = American National Standards Institute \\
ISO = International Standards Organization}

\footer
