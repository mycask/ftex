\section{The \texorpdfstring{\inlinefortranss{if}}{if} Statements}

An important part of any programming language are the
\textit{conditional statements}. The most common such statement in
Fortran is the \inlinefortranss{if} statement, which actually has
several forms. The simplest one is the logical \inlinefortranss{if}
statement:

\begin{fortran77}
      if (<<logical expression>>) <<executable statement>>
\end{fortran77}

This has to be written on one line. This example finds the absolute
value of \inlinefortranss{x}:

\begin{fortran77}
      if (x .LT. 0) x = -x
\end{fortran77}

If more than one statement should be executed inside the
\inlinefortranss{if}, then the following syntax should be used:

\begin{fortran77}
      if (<<logical expression>>) then
         <<statements>>
      endif
\end{fortran77}

The most general form of the \inlinefortranss{if} statement has the
following form:

\begin{fortran77}
      if (<<logical expression>>) then
         <<statements>>
      elseif (<<logical expression>>) then
         <<statements>>
       :
       :
      else
         <<statements>>
      endif
\end{fortran77}

The execution flow is from top to bottom. The conditional expressions
are evaluated in sequence until one is found to be true. Then the
associated statements are executed and the control resumes after the
\inlinefortranss{endif}.


\subsection*{Nested \inlinefortranss{if} statements}

\inlinefortranss{if} statements can be nested in several levels. To
ensure readability, it is important to use proper indentation. Here is
an example:

\begin{fortran77}
      if (x .GT. 0) then
         if (x .GE. y) then
            write(*,*) 'x is positive and x >= y'
         else
            write(*,*) 'x is positive but x < y'
         endif
      elseif (x .LT. 0) then
         write(*,*) 'x is negative'
      else
         write(*,*) 'x is zero'
      endif
\end{fortran77}

You should avoid nesting many levels of \inlinefortranss{if} statements
since things get hard to follow.
