\section{How to Use Fortran 90 on the Unix Computers at Stanford}

\subsection*{Practical Details}

Unfortunately, the Fortran 90 compiler that resides on the Sun computers
in the leland system is unreliable. However, there is no alternative at
this time. If you think your program should compile but it won't, then
it might not be your fault. See the instructor if you think the compiler
is misbehaving.


\subsection*{Compiling, Linking, and Executing}

In virtually the same manner in which we compiled our Fortran 77
routines, we can process our Fortran 90 programs. To compile and link a
Fortran 90 source code with the name \inlinecode{main.f90}, simply type

\begin{code}
f90 main.f90
\end{code}

This will create an executable file called \inlinecode{a.out}.

Fortran 90 programs which make use of the free format form (to be
explained in the next section) must have an extension \inlinecode{.f90}.

Just as with the Fortran 77 compiler, we are able to specify the name of
our executable file with the Fortran90 compiler by specifying the
\inlinecode{-o} option:

\begin{code}
f90 main.f90 -o main.out
\end{code}

This will create an executable file called \inlinecode{main.out}.

Similarly, we can link to a Fortran 90 compiled library (such as the
BLAS/LAPACK library in the class account) by

\begin{code}
f90 main.f90 -L/usr/class/me390/lib -lmy_lib90
\end{code}

This will link your program \inlinecode{main.f90} to the compiled
library called \inlinecode{libmy_lib90.a} which resides on the class
account in the directory \ifmini{\\}{}\inlinecode{/usr/class/me390/lib}.
You may copy this library file to your own account if you wish. However,
the file is rather large and may cause you to exceed your disk quota on
the leland system. A better approach would be to simply compile your
programs requiring the BLAS/LAPACK library using the command given
above.

Note: the library \inlinecode{libmy_lib90.a} MUST have been created
using the Fortran 90 compiler. Attempts to link a Fortran 90 program to
a Fortran 77 library will fail.
