\section{Main Features of Fortran 90}

Fortran 90 is a new programming language intended for use in scientific
and engineering applications. It is a language that has developed by the
introduction of features that are new to Fortran, but are based on
experience of other languages (like C and Matlab for instance). Fortran
90 is very different from earlier versions of Fortran, yet it is
completely backwards compatible with Fortran 77.

The features of Fortran 90 are far too numerous to mention in entirety
here, but some of the key features are outlined below:

\newcommand{\descitem}[2]{\item[\normalfont{}#1] \hfill \\ #2}
\begin{description}
    \descitem{Free format on source code.}{
        In Fortran 90, you can use either the Fortran 77 input format or
        free format. If you use free format, the file extension
        \inlinecode{.f90} should be used for the file name.
    }

    \descitem{Dynamic allocation and pointers.}{
        It is now possible to allocate storage dynamically. This enables
        us to finally get rid of all the ``work'' arrays!
    }

    \descitem{User defined data types.}{
        You can now define your own composite data types, similar to
        \inlinekeyword{struct} in C or \inlinekeyword{record} in Pascal.
    }

    \descitem{Modules.}{
        Modules enables you to program in an object oriented style,
        similar to C++. Modules can also be used to hide global
        variables, thereby making the Fortran 77 common construct
        outdated.
    }

    \descitem{Recursive functions.}{
        Now a part of the language.
    }

    \descitem{Built-in array operations.}{
        Statements like \inlinefortrann{A=0} and \inlinefortrann{C=A+B}
        are now valid when \inlinefortrann{A} and \inlinefortrann{B} are
        arrays. There are also built-in functions for matrix operations,
        e.g., matmul for performing matrix multiplication.
    }

    \descitem{Operator overloading.}{
        You can define your own meaning of operators like
        \inlinefortrann{+} and \inlinefortrann{=} for your own data
        types (objects).
    }
\end{description}
